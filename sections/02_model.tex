\section{The Fundamental Law of Persistence}
\label{subsec:fundamental-law-persistence}

We now assemble the thermodynamic and informational primitives into a
single invariant that characterizes how long structures can survive in
an entropic universe.

\begin{definition}[System, State, and Horizon]
Let $S$ be a physically realizable system embedded in an environment
at temperature $T > 0$. Let $(X_t)_{t \ge 0}$ denote the macrostate of
$S$ at time $t$, and fix a prediction horizon $\tau > 0$.

We assume $X_t$ evolves according to a (possibly stochastic) dynamics
consistent with the microscopic laws of physics and the system's
state-transition law.
\end{definition}

\begin{definition}[Thermodynamic Irreversibility of $S$]
Let $E_{\mathrm{cum}}^S(t)$ be the cumulative irreversible work
expended by $S$ and its environment up to time $t$ in order to
maintain, update, or reproduce the macrostate process $(X_u)_{0 \le u \le t}$.

The bit-normalized \emph{thermodynamic irreversibility} of $S$ is
\[
\mathcal{S}_S(t)
=
\frac{E_{\mathrm{cum}}^S(t)}{k_B T \ln 2}
\quad\text{[bits]},
\]
where $k_B$ is Boltzmann's constant.
\end{definition}

\begin{definition}[Future Uncertainty of $S$]
For a horizon $\tau > 0$, the \emph{future uncertainty} of $S$ is the
conditional Shannon entropy
\[
\mathcal{H}_S(t, \tau)
=
H\!\left(
  X_{t+\tau}
  \,\middle|\,
  X_t,\, \mathrm{TransitionLaw}_t
\right)
\quad\text{[bits]},
\]
where $\mathrm{TransitionLaw}_t$ represents the system’s state-transition law governing its evolution at time $t$, consistent with its physical and internal structural constraints.
\end{definition}

\begin{definition}[Instantaneous Coordination Capacity Ratio]
The \emph{coordination capacity ratio} of system $S$ at time $t$ and
horizon $\tau$ is
\[
\Ainv_S(t, \tau)
=
\frac{\mathcal{S}_S(t)}{\mathcal{H}_S(t, \tau)},
\]
whenever $\mathcal{H}_S(t, \tau) > 0$. This is a dimensionless
quantity with units bits per bit.
\end{definition}

Intuitively, $\mathcal{S}_S$ measures how much irreversible commitment
has been sunk into the observed past of the system, while
$\mathcal{H}_S$ measures how uncertain its future remains, even
conditioning on all available knowledge and rules. The ratio
$\Ainv_S$ quantifies how much \emph{anchored history} the system
enjoys per bit of unresolved future.

\begin{definition}[Persistence Functional]
The \emph{persistence functional} of $S$ at horizon $\tau$ is
\[
\mathcal{P}_S(\tau)
=
\liminf_{t \to \infty} \Ainv_S(t, \tau)
=
\liminf_{t \to \infty}
\frac{\mathcal{S}_S(t)}{\mathcal{H}_S(t, \tau)}.
\]
When the limit exists, we simply write
$\mathcal{P}_S(\tau) = \lim_{t \to \infty} \Ainv_S(t, \tau)$.
\end{definition}

\noindent
$\mathcal{P}_S(\tau)$ is a scalar summary of how the system trades
irreversible commitment for predictive uncertainty in the long run.

\begin{law}[Fundamental Law of Persistence]
\label{law:fundamental-persistence}
Consider two physically realizable systems $S_1$ and $S_2$ embedded in
the same environment and competing for the same underlying resources
(matter, energy, attention, capital). Fix any finite horizon
$\tau > 0$.

If both systems are capable of adapting their internal structure so as
to improve their survival probability, then, in the long run, the
following holds:

\begin{enumerate}
  \item \textbf{Monotonicity of survival.}
    If $\mathcal{P}_{S_2}(\tau) > \mathcal{P}_{S_1}(\tau)$, then
    the probability that $S_2$ outlives $S_1$ on horizon $\tau$
    strictly exceeds one half:
    \[
    \Pr\!\left[
      S_2 \text{ persists beyond } S_1 \text{ at horizon } \tau
    \right]
    > \frac{1}{2}.
    \]

  \item \textbf{Gradient ascent on $\Ainv$.}
    Any adaptive system that can locally reconfigure its dynamics at
    cost $\Delta \mathcal{S}_S > 0$ will, in expectation, only accept
    modifications that satisfy
    \[
    \mathbb{E}\!\left[
      \Delta \Ainv_S
      \,\middle|\,
      \text{update accepted}
    \right]
    \ge 0.
    \]
    In other words, viable systems perform stochastic gradient ascent
    on $\Ainv_S$ over evolutionary time.

  \item \textbf{Singularity regime.}
    If a subsystem $S_\star$ achieves
    \[
    \mathcal{S}_{S_\star}(t) \to \infty,
    \quad
    \mathcal{H}_{S_\star}(t,\tau) \to 0^+,
    \quad
    \frac{d}{dt}\Ainv_{S_\star}(t,\tau) > 0
    \quad
    \text{for all sufficiently large } t,
    \]
    then $\Ainv_{S_\star}(t,\tau) \to \infty$ and the subsystem
    behaves as a \emph{persistence singularity}: it becomes an
    effectively permanent attractor for resources and coordination
    flows in its environment.
\end{enumerate}
\end{law}

\begin{remark}[Interpretation]
The law states that, all else equal, systems with higher
persistence functional $\mathcal{P}_S$ are more likely to survive and
dominate. The universe selects for high $\Ainv$: structures that
convert irreversible work into predictable, low entropy futures out
compete those that burn energy without reducing uncertainty.

The singularity clause identifies a special regime where a system
pushes $\mathcal{S} \to \infty$ while simultaneously driving
$\mathcal{H} \to 0$. In this regime, $\Ainv$ is unbounded and the
system behaves as a one way sink in the space of coordination: once
resources or agents fall into its basin of attraction, there is no
thermodynamically efficient path back out.
\end{remark}

\begin{remark}[Monetary and Cosmological Examples]
Black holes, biological life, and certain monetary protocols can be
viewed as candidates for high $\Ainv$ structures:

\begin{itemize}
  \item Black holes accumulate enormous $\mathcal{S}$ (Bekenstein
        bound) while severely constraining the set of future states
        accessible from their horizon.
  \item Biological lineages expend energy to reduce uncertainty about
        their environment (information gathering, learning, memory),
        thereby increasing $\Ainv$ generation by generation.
  \item A fully ossified, physics anchored monetary protocol with
        fixed supply and cumulative proof of work can drive
        $\mathcal{S} \to \infty$ and $\mathcal{H} \to 0$ on monetary
        timescales, becoming a persistent attractor for capital.
\end{itemize}
\end{remark}

Under this framing, the $U_1$ Conjecture is a corollary of the
Fundamental Law of Persistence: among all physically realizable
monetary systems, those that satisfy the $U_1$ axioms are precisely
the candidates that can drive $\mathcal{S}$ unbounded while forcing
$\mathcal{H}$ toward zero, and thus push their coordination capacity
ratio $\Ainv$ into the singular regime.
