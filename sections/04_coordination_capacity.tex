\section{Coordination Capacity}

This section formalizes the notion of \emph{coordination capacity} for open,
adversarial monetary systems, isolates irreversibility as the primitive
resource, and states the $U_1$ Coordination Capacity Conjecture.


\subsection{Irreversibility as the Primitive}

\begin{definition}[Irreversible Work Step]
A work step is \emph{irreversible} if its physical realization increases
entropy in the environment such that no observer with bounded resources
can algorithmically reconstruct a pre-image of the microscopic state
prior to the step.

We write $E_{\irr} > 0$ for the minimal energy cost of such a step at
a given ambient temperature.
\end{definition}

\begin{definition}[Physics-Anchored Protocol]
A protocol $\Pi$ is \emph{physics-anchored} if every admissible block
production event must include at least one irreversible work step
whose successful completion is:
\begin{enumerate}
  \item publicly and efficiently verifiable, and
  \item infeasible to simulate without incurring comparable
        physical energy cost.
\end{enumerate}
\end{definition}

In other words, physics-anchored protocols encode consensus progress
into the irreversibility of the underlying physical process (e.g. PoW),
rather than into institutional decree or access-controlled identities.


\subsection{Atomic Unit of Account}

\begin{definition}[Atomic Unit of Account]
Let $\Pi$ be a physics-anchored monetary protocol with total supply
process $(U_t)_{t \ge 0}$.

An \emph{atomic unit of account} $U$ is the smallest denomination such that:
\begin{enumerate}
  \item (Indivisibility) No valid ledger state can represent a claim
        smaller than $U$.
  \item (Fungibility) Any two claims of size $U$ are mutually
        substitutable under the protocol's transfer rules.
  \item (Monetary Closure) The aggregate supply satisfies
        $\sum U_t \le S_{\max}$ across all valid histories.
  \item (Settlement Granularity) Every irreversible settlement
        event can be decomposed into a finite change in the allocation
        of atomic units $U$.
\end{enumerate}
\end{definition}

\begin{remark}
In Bitcoin, the atomic unit is $U = 1$ satoshi and $S_{\max} = 21 \times
10^6 \text{ BTC} = 2.1 \times 10^{15}$ satoshis. For any protocol
satisfying $A_4$ (Monetary Closure), an analogous $U$ exists.
\end{remark}


\subsection{Coordination Flow and Capacity}

We now formalize coordination as an information flow from latent
economic reality into a publicly settled ledger.

\begin{definition}[Economic State and Ledger State]
Let $(\Theta_t)_{t \ge 0}$ denote a (possibly high-dimensional)
stochastic process capturing the true economic preferences and claims
of agents at time $t$, and let $(A_t)_{t \ge 0}$ denote the publicly
settled ledger state produced by protocol $\Pi$ at time $t$.

Both processes are defined on a common probability space, and
$(A_t)$ is adapted to the filtration generated by protocol-visible
messages and blocks.
\end{definition}

\begin{definition}[Coordination Flow]
For a fixed protocol $\Pi$, the \emph{coordination flow rate} is
\[
\dot V_{\coord}(\Pi)
=
\lim_{T \to \infty}
\frac{1}{T}\, I(\Theta_{0:T}; A_{0:T}),
\]
whenever the limit exists, where $I(\cdot;\cdot)$ denotes mutual
information.
\end{definition}

Intuitively, $\dot V_{\coord}(\Pi)$ measures how many bits of true
economic intent are irreversibly compressed into the public ledger per
unit time.

\begin{definition}[Coordination Capacity]
Let $\mathcal{M}_{\phys}$ denote the class of physics-anchored protocols
satisfying:
\begin{enumerate}
  \item public verifiability ($A_2$),
  \item open participation (permissionless entry and exit),
  \item bounded adversary resources (e.g.\ an honest-majority work budget),
  \item physically irreversible settlement cost ($A_1$).
\end{enumerate}

The \emph{coordination capacity} of the environment is
\[
C_{\coord}
=
\sup_{\Pi \in \mathcal{M}_{\phys}}
\dot V_{\coord}(\Pi)
=
\sup_{\Pi \in \mathcal{M}_{\phys}}
\lim_{T \to \infty}
\frac{1}{T}\, I(\Theta_{0:T}; A_{0:T}).
\]
\end{definition}

Thus $C_{\coord}$ is to adversarial economic settlement what Shannon
capacity is to communication over noisy channels: the maximal
information rate at which true economic state can be mapped into a
canonical public ledger with arbitrarily small corruption probability,
under fixed physical and adversarial constraints.


\subsection{Security Exponent and Irreversibility Density}

We now connect settlement reliability to the rate of irreversible work
performed by honest participants.

\begin{definition}[Security Exponent]
Consider a physics-anchored protocol $\Pi$ exhibiting probabilistic
finality ($A_3$). For a reorganization depth parameter $k \in \mathbb{N}$,
let
\[
P_{\reorg}(k) = \Pr(\text{ledger reorganization of depth} \ge k).
\]

If there exists $\beta > 0$ and $k_0$ such that
\[
P_{\reorg}(k) \le \exp(-\beta k)
\quad
\text{for all } k \ge k_0,
\]
we say that $\Pi$ has \emph{security exponent} $\beta$.
\end{definition}

\begin{definition}[Irreversibility Density]
Let $\dot E_{\honest}$ denote the long-run honest irreversible work
rate (in joules per second) invested in maintaining consensus for
$\Pi$, and let $\dot U_{\secured}$ denote the long-run rate (in units
per second) at which atomic units $U$ are involved in economically
meaningful settled transfers.

The \emph{irreversibility density} of $\Pi$ is
\[
\rho_{\irr}(\Pi)
=
\frac{\dot E_{\honest}}{\dot U_{\secured}}
\quad
\text{with units}
\quad
\frac{\mathrm{J}}{\mathrm{U}}.
\]
\end{definition}

\begin{remark}
Heuristically, $\rho_{\irr}$ can be interpreted as the average amount
of physically irreversible work crystallized into the ledger per unit
of account. Systems with $\rho_{\irr} \to 0$ cannot maintain a
meaningful barrier against adversarial rewriting of history in an
open, permissionless setting.
\end{remark}

In PoW-style $U_1$ systems, empirical and theoretical arguments suggest
that, under competitive mining and difficulty adjustment ($T_1$, $T_2$),
there exists a protocol-dependent constant $c > 0$ such that
\[
\beta(\Pi) \approx c \cdot
\frac{\dot E_{\honest}}{V_{\secured}},
\]
where $V_{\secured}$ is the total economic value (e.g.\ in units $U$
times purchasing power) that depends on $\Pi$ for settlement.\footnote{
The precise choice of $V_{\secured}$ (e.g.\ fee-paying volume, non-churn
flow, or broader balance at risk) is an empirical modeling decision.
For the purposes of the conjecture, it suffices that $V_{\secured}$
captures the value whose coordination depends critically on $\Pi$.}


\subsection{Model Boundary: Physics vs.\ Coercion}

The above definitions explicitly restrict attention to
\emph{physics-anchored} protocols in $\mathcal{M}_{\phys}$.

\begin{definition}[Institution-Anchored Ledger]
A ledger $\Pi_{\mathrm{inst}}$ is \emph{institution-anchored} if irreversibility
of entries is enforced primarily by:
\begin{enumerate}
  \item legal decree or administrative authority,
  \item access-controlled identities or whitelists,
  \item coercive enforcement (e.g.\ courts, police, capital controls),
\end{enumerate}
rather than by publicly verifiable irreversible work steps.
\end{definition}

Fiat currencies, bank databases, and permissioned ledgers typically
fall into this class: they may exhibit practical irreversibility within
a jurisdiction, but it is not rooted in physics and is not
permissionless or globally auditable in the sense of $A_1$–$A_4$.

\begin{remark}
Coordination capacity $C_{\coord}$ is defined only over
$\mathcal{M}_{\phys}$. Institution-anchored systems may coordinate
agents via coercive trust hierarchies, but they do not approach a
physics-limited capacity in the Shannon sense, and they do not provide
irreversibility that is independent of specific institutional
arrangements.
\end{remark}


\subsection{\texorpdfstring{$U_1$ Coordination Capacity Conjecture}{U1 Coordination Capacity Conjecture}}

We can now restate the core conjecture in this refined language.

\begin{conjecture}[$U_1$ Coordination Capacity Conjecture]
Consider the class $\mathcal{M}_{\phys}$ of physics-anchored protocols
operating in an open, permissionless, adversarial environment with a
bounded-work attacker (honest-majority assumption).

Then:
\begin{enumerate}
  \item A protocol $\Pi \in \mathcal{M}_{\phys}$ achieves strictly
        positive coordination capacity ($\dot V_{\coord}(\Pi) > 0$)
        if and only if it satisfies the $U_1$ axioms:
        $A_1$ (energy-bounded issuance),
        $A_2$ (public verifiability),
        $A_3$ (probabilistic finality),
        $A_4$ (monetary closure).

  \item For any such $\Pi$ with adversary work fraction $q < p$
        (honest fraction $p$), there exists $\beta(\Pi) > 0$ such that
        the reorganization probability obeys an exponential bound
        \[
        \Pr(\mathrm{reorg\ depth} \ge k)
        \;\le\;
        \left(\frac{q}{p}\right)^k
        =
        \exp(-\beta(\Pi)\, k)
        \quad \text{for all sufficiently large } k.
        \]

  \item Among protocols satisfying $A_1$–$A_4$, $U_1$-style PoW monetary systems
        can achieve security exponents $\beta(\Pi)$ arbitrarily close
        to the physical limit imposed by their irreversibility density:
        \[
        \beta(\Pi)
        \lesssim
        c \cdot
        \frac{\dot E_{\honest}}{V_{\secured}},
        \]
        for some protocol- and environment-dependent constant
        $c > 0$, with equality approached in the limit of optimal
        encoding of economic state into ledger state.
\end{enumerate}
\end{conjecture}

\begin{corollary}[Vanishing Irreversibility Density]
Within $\mathcal{M}_{\phys}$, any family of protocols
$(\Pi_n)_{n \in \mathbb{N}}$ with irreversibility density
$\rho_{\irr}(\Pi_n) \to 0$ as $n \to \infty$ satisfies
\[
\lim_{n \to \infty} \dot V_{\coord}(\Pi_n) = 0.
\]
In particular, physics-anchored systems whose marginal unit of account
is not backed by a nontrivial amount of irreversible work cannot
achieve nonzero coordination capacity in the limit.
\end{corollary}

The conjecture thus states that $U_1$ systems are to economic coordination
what Shannon-capacity-achieving codes are to communication: they
saturate a physically imposed limit on reliable performance in a noisy,
adversarial universe.

\subsection{Dual-Entropy Characterization}

We now introduce a dimensionless measure that unifies thermodynamic
irreversibility with informational predictability. This characterization
provides a scalar invariant that captures the essential tension in
monetary coordination: certainty about the past versus uncertainty
about the future.


\subsubsection{Thermodynamic Irreversibility}

\begin{definition}[Thermodynamic Irreversibility (Bit-Normalized)]
\label{def:thermo-entropy}
For a physics-anchored protocol $\Pi$ with cumulative irreversible
energy expenditure $E_{\mathrm{cum}}(t)$ up to time $t$, define the
\emph{thermodynamic irreversibility} (in bits) as
\[
\mathcal{S}_\Pi(t)
\;=\;
\frac{E_{\mathrm{cum}}(t)}{k_B T \ln 2},
\]
where $k_B \approx 1.38 \times 10^{-23}\;\mathrm{J/K}$ is Boltzmann's
constant, $T$ is ambient temperature (typically $\approx 300\;\mathrm{K}$),
and $k_B T \ln 2 \approx 2.87 \times 10^{-21}\;\mathrm{J}$ is the
Landauer energy cost per bit of erasure.
\end{definition}

\begin{remark}
$\mathcal{S}_\Pi$ counts the number of bits whose erasure would require
energy equal to the cumulative work invested in $\Pi$. This provides a
physics-grounded measure of how ``deep'' the protocol's history is
anchored in thermodynamic irreversibility. The Landauer normalization
ensures $\mathcal{S}$ has units of bits, enabling direct comparison
with informational entropy.
\end{remark}

\begin{remark}[Connection to Landauer's Principle]
Landauer's principle (1961) establishes that erasing one bit of
information requires dissipating at least $k_B T \ln 2$ joules of
energy as heat. By normalizing cumulative work by this quantity,
$\mathcal{S}_\Pi$ measures the thermodynamic ``depth'' of the ledger
in fundamental physical units.
\end{remark}


\subsubsection{Monetary Informational Entropy}

\begin{definition}[Monetary Informational Entropy]
\label{def:info-entropy}
For a monetary protocol $\Pi$ with supply process $(U_t)_{t \ge 0}$
and policy state $(\theta_t)_{t \ge 0}$, define the \emph{monetary
informational entropy} at horizon $\tau > 0$ as
\[
\mathcal{H}_\Pi(\tau)
\;=\;
H\bigl(\mathrm{MonetaryState}_{t+\tau} \,\big|\,
       \mathrm{MonetaryState}_t,\, \mathrm{Protocol}_t\bigr),
\]
where $H(\cdot \mid \cdot)$ denotes conditional Shannon entropy
(in bits), and $\mathrm{MonetaryState}$ encompasses:
\begin{itemize}
  \item total supply $U_t$,
  \item issuance rate $\dot{U}_t$,
  \item policy parameters (inflation targets, governance rules),
  \item any state variables affecting future monetary outcomes.
\end{itemize}
\end{definition}

\begin{remark}
$\mathcal{H}_\Pi(\tau)$ measures the irreducible uncertainty (in bits)
about the monetary system's future state at horizon $\tau$, given
complete knowledge of its current state and rules. For protocols with
deterministic supply schedules and ossified rule sets,
$\mathcal{H}_\Pi(\tau) \to 0$ for all $\tau$.
\end{remark}

\begin{remark}[Orthogonality of $\mathcal{S}$ and $\mathcal{H}$]
These two quantities are logically independent:
\begin{itemize}
  \item $\mathcal{S}$ measures \emph{how hard it is to change the past}
        (thermodynamic irreversibility).
  \item $\mathcal{H}$ measures \emph{how hard it is to predict the future}
        (informational uncertainty).
\end{itemize}
A system can have any combination: high $\mathcal{S}$ with high
$\mathcal{H}$ (irreversible but unpredictable), low $\mathcal{S}$
with low $\mathcal{H}$ (reversible but predictable), etc.
\end{remark}


\subsubsection{The Coordination Capacity Ratio}

\begin{definition}[Coordination Capacity Ratio]
\label{def:Ainv}
The \emph{coordination capacity ratio} of protocol $\Pi$ is
\[
\boxed{
\Ainv_\Pi
\;\equiv\;
\frac{\mathcal{S}_\Pi}{\mathcal{H}_\Pi}
}
\]
with units: $\mathrm{bits} / \mathrm{bits} = \text{dimensionless}$.
\end{definition}

\begin{interpretation}
$\Ainv_\Pi$ measures the ratio of thermodynamic commitment (secured
past) to monetary uncertainty (unknown future):
\[
\Ainv = \frac{\text{Irreversibility}}{\text{Uncertainty}}
       = \frac{\text{What cannot be undone}}{\text{What cannot be known}}.
\]
High $\Ainv$ indicates a system with strong irreversibility and high
predictability---the optimal configuration for coordination.
\end{interpretation}

\begin{remark}[Analogy to Fundamental Dimensionless Ratios]
The ratio $\Ainv = \mathcal{S}/\mathcal{H}$ is structurally analogous
to other fundamental dimensionless quantities in physics:
\begin{center}
\begin{tabular}{lll}
\textbf{Ratio} & \textbf{Formula} & \textbf{Tension} \\
\hline
Reynolds number & $Re = \rho v L / \mu$ & Inertia vs.\ viscosity \\
Signal-to-noise & $\mathrm{SNR} = P_s / P_n$ & Signal vs.\ noise \\
Carnot efficiency & $\eta = 1 - T_C/T_H$ & Cold vs.\ hot reservoir \\
\textbf{Coordination capacity} & $\Ainv = \mathcal{S}/\mathcal{H}$ &
  \textbf{Irreversibility vs.\ uncertainty}
\end{tabular}
\end{center}
\end{remark}


\subsubsection{Properties of \texorpdfstring{$\Ainv$}{A-inv}}

\begin{proposition}[Boundary Behavior]
\label{prop:Ainv-bounds}
For any physics-anchored protocol $\Pi$:
\begin{enumerate}
  \item If $\mathcal{S}_\Pi \to \infty$ and $\mathcal{H}_\Pi \to 0^+$,
        then $\Ainv_\Pi \to \infty$.
  \item If $\mathcal{S}_\Pi \to 0$ (no thermodynamic anchoring),
        then $\Ainv_\Pi \to 0$ regardless of $\mathcal{H}_\Pi$.
  \item If $\mathcal{H}_\Pi \to \infty$ (unbounded policy uncertainty),
        then $\Ainv_\Pi \to 0$ regardless of $\mathcal{S}_\Pi$.
\end{enumerate}
\end{proposition}

\begin{proof}
Direct from the definition $\Ainv = \mathcal{S}/\mathcal{H}$ and
properties of limits.
\end{proof}

\begin{corollary}[Optimal Coordination Regime]
Maximum coordination capacity is achieved in the regime:
\[
\mathcal{S} \to \infty, \quad \mathcal{H} \to 0
\quad \Longrightarrow \quad
\Ainv \to \infty.
\]
This corresponds to a protocol with maximal thermodynamic security
and minimal monetary uncertainty.
\end{corollary}


\subsubsection{Capital Flow Conjecture}

\begin{conjecture}[Capital Flow Gradient]
\label{conj:capital-flow}
Let $\Pi_A$ and $\Pi_B$ be competing monetary protocols with
coordination capacity ratios $\Ainv_A$ and $\Ainv_B$ respectively.

If $\Ainv_B > \Ainv_A$, then in the long run, capital flows from
$\Pi_A$ to $\Pi_B$ at a rate proportional to the gradient:
\[
\dot{K}_{A \to B}
\;\propto\;
(\Ainv_B - \Ainv_A) \cdot K_A,
\]
where $K_A$ is capital denominated in system $A$.
\end{conjecture}

\begin{remark}
This conjecture formalizes the intuition that capital flows ``downhill''
in entropy space, analogous to:
\begin{itemize}
  \item heat flow down temperature gradients (thermodynamics),
  \item mass flow down gravitational potentials (mechanics),
  \item current flow down voltage differentials (electromagnetism).
\end{itemize}
The coordination capacity differential $\Delta\Ainv = \Ainv_B - \Ainv_A$
acts as the ``potential gradient'' driving capital migration.
\end{remark}


\subsubsection{Empirical Estimates}

We now apply the framework to three monetary systems.

\begin{example}[Bitcoin]
\label{ex:Ainv-btc}
For Bitcoin as of 2025:
\begin{itemize}
  \item Cumulative PoW energy:
        $E_{\mathrm{cum}} \approx 2.7 \times 10^{18}\;\mathrm{J}$
        (estimated 750 TWh total since 2009).
  \item At $T = 300\;\mathrm{K}$:
        \[
        \mathcal{S}_{\mathrm{BTC}}
        = \frac{2.7 \times 10^{18}}{2.87 \times 10^{-21}}
        \approx 10^{39}\;\text{bits}.
        \]
  \item Supply uncertainty: $\mathcal{H}_{\mathrm{BTC}} \approx 1$ bit
        (deterministic halving schedule, protocol ossification,
        Schelling point against rule changes).
  \item Coordination capacity ratio:
        $\Ainv_{\mathrm{BTC}} \approx 10^{39}$.
\end{itemize}
\end{example}

\begin{example}[Gold]
\label{ex:Ainv-gold}
For gold as a monetary system:
\begin{itemize}
  \item Physical extraction energy provides atomic verification
        ($\sim 10^{19}$--$10^{20}$ bits equivalent) but does not
        secure the ownership ledger.
  \item Ownership record is institution-anchored (vulnerable to
        confiscation, rehypothecation, paper claims).
  \item Effective thermodynamic irreversibility of the \emph{monetary}
        system: $\mathcal{S}_{\mathrm{Gold}} \approx 10^{19}$--$10^{20}$
        bits (physical verification only).
  \item Policy uncertainty (confiscation risk, paper-to-physical ratio,
        export controls): $\mathcal{H}_{\mathrm{Gold}} \approx 10$ bits.
  \item Coordination capacity ratio:
        $\Ainv_{\mathrm{Gold}} \approx 10^{18}$--$10^{19}$.
\end{itemize}
\end{example}

\begin{example}[Fiat Currency (USD)]
\label{ex:Ainv-usd}
For the US dollar:
\begin{itemize}
  \item No thermodynamic anchoring of the ledger (database entries,
        editable by administrative action):
        $\mathcal{S}_{\mathrm{USD}} \approx 0$.
  \item High policy uncertainty (M2 growth unpredictable, inflation
        targets adjustable, fiscal policy volatile, CBDC transition):
        $\mathcal{H}_{\mathrm{USD}} \approx 20$ bits.
  \item Coordination capacity ratio:
        $\Ainv_{\mathrm{USD}} \approx 0$.
\end{itemize}
\end{example}

\begin{corollary}[Hierarchy of Coordination Capacity]
\label{cor:hierarchy}
Under the above estimates:
\[
\Ainv_{\mathrm{BTC}}
\;\gg\;
\Ainv_{\mathrm{Gold}}
\;\gg\;
\Ainv_{\mathrm{USD}}.
\]
Quantitatively:
\begin{align*}
\Ainv_{\mathrm{BTC}} / \Ainv_{\mathrm{Gold}}
  &\approx 10^{20}, \\
\Ainv_{\mathrm{Gold}} / \Ainv_{\mathrm{USD}}
  &\approx 10^{18}\text{--}10^{19}.
\end{align*}
The ratio $\Ainv_{\mathrm{BTC}} / \Ainv_{\mathrm{Gold}} \approx 10^{20}$
suggests Bitcoin represents a \emph{qualitative phase transition} in
coordination capacity, not merely an incremental improvement over
prior monetary technologies.
\end{corollary}


\subsubsection{Relation to Prior Definitions}

The coordination capacity ratio $\Ainv$ complements the earlier
definitions in this section:

\begin{itemize}
  \item \textbf{Irreversibility density} $\rho_{\irr}$ (Definition 4.X)
        measures joules per unit transferred---a flow quantity.
  \item \textbf{Security exponent} $\beta$ (Definition 4.X) measures
        the exponential decay rate of reorganization probability.
  \item \textbf{Coordination capacity ratio} $\Ainv$ measures the
        ratio of cumulative thermodynamic depth to monetary uncertainty---a
        stock quantity capturing the system's attractiveness as a
        coordination substrate.
\end{itemize}

These quantities are related: high $\rho_{\irr}$ contributes to high
$\mathcal{S}$ (cumulative irreversibility), while protocol ossification
contributes to low $\mathcal{H}$ (monetary predictability). Together,
they determine $\Ainv$.

\begin{remark}[Why $\Ainv$ Matters]
While $C_{\mathrm{coord}}$, $\rho_{\irr}$, and $\beta$ characterize
the \emph{operational} properties of a protocol (capacity, density,
security), $\Ainv$ characterizes its \emph{attractiveness} as a
coordination substrate. The Capital Flow Conjecture (Conjecture
\ref{conj:capital-flow}) posits that $\Ainv$ determines long-run
capital allocation across competing monetary systems.
\end{remark}
