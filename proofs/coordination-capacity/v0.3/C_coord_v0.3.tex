% C_coord_v0.3.tex
% Dual-Entropy Characterization of Coordination Capacity
% Author: Stamatis Petridis
% Date: December 2025
%
% This version introduces the dimensionless coordination capacity ratio
% kappa = S/H, unifying thermodynamic irreversibility with informational
% predictability.

\subsection{Dual-Entropy Characterization}

We now introduce a dimensionless measure that unifies thermodynamic
irreversibility with informational predictability. This characterization
provides a scalar invariant that captures the essential tension in
monetary coordination: certainty about the past versus uncertainty
about the future.


\subsubsection{Thermodynamic Irreversibility}

\begin{definition}[Thermodynamic Irreversibility (Bit-Normalized)]
\label{def:thermo-entropy}
For a physics-anchored protocol $\Pi$ with cumulative irreversible
energy expenditure $E_{\mathrm{cum}}(t)$ up to time $t$, define the
\emph{thermodynamic irreversibility} (in bits) as
\[
\mathcal{S}_\Pi(t)
\;=\;
\frac{E_{\mathrm{cum}}(t)}{k_B T \ln 2},
\]
where $k_B \approx 1.38 \times 10^{-23}\;\mathrm{J/K}$ is Boltzmann's
constant, $T$ is ambient temperature (typically $\approx 300\;\mathrm{K}$),
and $k_B T \ln 2 \approx 2.87 \times 10^{-21}\;\mathrm{J}$ is the
Landauer energy cost per bit of erasure.
\end{definition}

\begin{remark}
$\mathcal{S}_\Pi$ counts the number of bits whose erasure would require
energy equal to the cumulative work invested in $\Pi$. This provides a
physics-grounded measure of how ``deep'' the protocol's history is
anchored in thermodynamic irreversibility. The Landauer normalization
ensures $\mathcal{S}$ has units of bits, enabling direct comparison
with informational entropy.
\end{remark}

\begin{remark}[Connection to Landauer's Principle]
Landauer's principle (1961) establishes that erasing one bit of
information requires dissipating at least $k_B T \ln 2$ joules of
energy as heat. By normalizing cumulative work by this quantity,
$\mathcal{S}_\Pi$ measures the thermodynamic ``depth'' of the ledger
in fundamental physical units.
\end{remark}


\subsubsection{Monetary Informational Entropy}

\begin{definition}[Monetary Informational Entropy]
\label{def:info-entropy}
For a monetary protocol $\Pi$ with supply process $(U_t)_{t \ge 0}$
and policy state $(\theta_t)_{t \ge 0}$, define the \emph{monetary
informational entropy} at horizon $\tau > 0$ as
\[
\mathcal{H}_\Pi(\tau)
\;=\;
H\bigl(\mathrm{MonetaryState}_{t+\tau} \,\big|\,
       \mathrm{MonetaryState}_t,\, \mathrm{Protocol}_t\bigr),
\]
where $H(\cdot \mid \cdot)$ denotes conditional Shannon entropy
(in bits), and $\mathrm{MonetaryState}$ encompasses:
\begin{itemize}
  \item total supply $U_t$,
  \item issuance rate $\dot{U}_t$,
  \item policy parameters (inflation targets, governance rules),
  \item any state variables affecting future monetary outcomes.
\end{itemize}
\end{definition}

\begin{remark}
$\mathcal{H}_\Pi(\tau)$ measures the irreducible uncertainty (in bits)
about the monetary system's future state at horizon $\tau$, given
complete knowledge of its current state and rules. For protocols with
deterministic supply schedules and ossified rule sets,
$\mathcal{H}_\Pi(\tau) \to 0$ for all $\tau$.
\end{remark}

\begin{remark}[Orthogonality of $\mathcal{S}$ and $\mathcal{H}$]
These two quantities are logically independent:
\begin{itemize}
  \item $\mathcal{S}$ measures \emph{how hard it is to change the past}
        (thermodynamic irreversibility).
  \item $\mathcal{H}$ measures \emph{how hard it is to predict the future}
        (informational uncertainty).
\end{itemize}
A system can have any combination: high $\mathcal{S}$ with high
$\mathcal{H}$ (irreversible but unpredictable), low $\mathcal{S}$
with low $\mathcal{H}$ (reversible but predictable), etc.
\end{remark}


\subsubsection{The Coordination Capacity Ratio}

\begin{definition}[Coordination Capacity Ratio]
\label{def:kappa}
The \emph{coordination capacity ratio} of protocol $\Pi$ is
\[
\boxed{
\kappa_\Pi
\;\equiv\;
\frac{\mathcal{S}_\Pi}{\mathcal{H}_\Pi}
}
\]
with units: $\mathrm{bits} / \mathrm{bits} = \text{dimensionless}$.
\end{definition}

\begin{interpretation}
$\kappa_\Pi$ measures the ratio of thermodynamic commitment (secured
past) to monetary uncertainty (unknown future):
\[
\kappa = \frac{\text{Irreversibility}}{\text{Uncertainty}}
       = \frac{\text{What cannot be undone}}{\text{What cannot be known}}.
\]
High $\kappa$ indicates a system with strong irreversibility and high
predictability---the optimal configuration for coordination.
\end{interpretation}

\begin{remark}[Analogy to Fundamental Dimensionless Ratios]
The ratio $\kappa = \mathcal{S}/\mathcal{H}$ is structurally analogous
to other fundamental dimensionless quantities in physics:
\begin{center}
\begin{tabular}{lll}
\textbf{Ratio} & \textbf{Formula} & \textbf{Tension} \\
\hline
Reynolds number & $Re = \rho v L / \mu$ & Inertia vs.\ viscosity \\
Signal-to-noise & $\mathrm{SNR} = P_s / P_n$ & Signal vs.\ noise \\
Carnot efficiency & $\eta = 1 - T_C/T_H$ & Cold vs.\ hot reservoir \\
\textbf{Coordination capacity} & $\kappa = \mathcal{S}/\mathcal{H}$ &
  \textbf{Irreversibility vs.\ uncertainty}
\end{tabular}
\end{center}
\end{remark}


\subsubsection{Properties of $\kappa$}

\begin{proposition}[Boundary Behavior]
\label{prop:kappa-bounds}
For any physics-anchored protocol $\Pi$:
\begin{enumerate}
  \item If $\mathcal{S}_\Pi \to \infty$ and $\mathcal{H}_\Pi \to 0^+$,
        then $\kappa_\Pi \to \infty$.
  \item If $\mathcal{S}_\Pi \to 0$ (no thermodynamic anchoring),
        then $\kappa_\Pi \to 0$ regardless of $\mathcal{H}_\Pi$.
  \item If $\mathcal{H}_\Pi \to \infty$ (unbounded policy uncertainty),
        then $\kappa_\Pi \to 0$ regardless of $\mathcal{S}_\Pi$.
\end{enumerate}
\end{proposition}

\begin{proof}
Direct from the definition $\kappa = \mathcal{S}/\mathcal{H}$ and
properties of limits.
\end{proof}

\begin{corollary}[Optimal Coordination Regime]
Maximum coordination capacity is achieved in the regime:
\[
\mathcal{S} \to \infty, \quad \mathcal{H} \to 0
\quad \Longrightarrow \quad
\kappa \to \infty.
\]
This corresponds to a protocol with maximal thermodynamic security
and minimal monetary uncertainty.
\end{corollary}


\subsubsection{Capital Flow Conjecture}

\begin{conjecture}[Capital Flow Gradient]
\label{conj:capital-flow}
Let $\Pi_A$ and $\Pi_B$ be competing monetary protocols with
coordination capacity ratios $\kappa_A$ and $\kappa_B$ respectively.

If $\kappa_B > \kappa_A$, then in the long run, capital flows from
$\Pi_A$ to $\Pi_B$ at a rate proportional to the gradient:
\[
\dot{K}_{A \to B}
\;\propto\;
(\kappa_B - \kappa_A) \cdot K_A,
\]
where $K_A$ is capital denominated in system $A$.
\end{conjecture}

\begin{remark}
This conjecture formalizes the intuition that capital flows ``downhill''
in entropy space, analogous to:
\begin{itemize}
  \item heat flow down temperature gradients (thermodynamics),
  \item mass flow down gravitational potentials (mechanics),
  \item current flow down voltage differentials (electromagnetism).
\end{itemize}
The coordination capacity differential $\Delta\kappa = \kappa_B - \kappa_A$
acts as the ``potential gradient'' driving capital migration.
\end{remark}


\subsubsection{Empirical Estimates}

We now apply the framework to three monetary systems.

\begin{example}[Bitcoin]
\label{ex:kappa-btc}
For Bitcoin as of 2025:
\begin{itemize}
  \item Cumulative PoW energy:
        $E_{\mathrm{cum}} \approx 2.7 \times 10^{18}\;\mathrm{J}$
        (estimated 750 TWh total since 2009).
  \item At $T = 300\;\mathrm{K}$:
        \[
        \mathcal{S}_{\mathrm{BTC}}
        = \frac{2.7 \times 10^{18}}{2.87 \times 10^{-21}}
        \approx 10^{39}\;\text{bits}.
        \]
  \item Supply uncertainty: $\mathcal{H}_{\mathrm{BTC}} \approx 1$ bit
        (deterministic halving schedule, protocol ossification,
        Schelling point against rule changes).
  \item Coordination capacity ratio:
        $\kappa_{\mathrm{BTC}} \approx 10^{39}$.
\end{itemize}
\end{example}

\begin{example}[Gold]
\label{ex:kappa-gold}
For gold as a monetary system:
\begin{itemize}
  \item Physical extraction energy provides atomic verification
        ($\sim 10^{19}$--$10^{20}$ bits equivalent) but does not
        secure the ownership ledger.
  \item Ownership record is institution-anchored (vulnerable to
        confiscation, rehypothecation, paper claims).
  \item Effective thermodynamic irreversibility of the \emph{monetary}
        system: $\mathcal{S}_{\mathrm{Gold}} \approx 10^{19}$--$10^{20}$
        bits (physical verification only).
  \item Policy uncertainty (confiscation risk, paper-to-physical ratio,
        export controls): $\mathcal{H}_{\mathrm{Gold}} \approx 10$ bits.
  \item Coordination capacity ratio:
        $\kappa_{\mathrm{Gold}} \approx 10^{18}$--$10^{19}$.
\end{itemize}
\end{example}

\begin{example}[Fiat Currency (USD)]
\label{ex:kappa-usd}
For the US dollar:
\begin{itemize}
  \item No thermodynamic anchoring of the ledger (database entries,
        editable by administrative action):
        $\mathcal{S}_{\mathrm{USD}} \approx 0$.
  \item High policy uncertainty (M2 growth unpredictable, inflation
        targets adjustable, fiscal policy volatile, CBDC transition):
        $\mathcal{H}_{\mathrm{USD}} \approx 20$ bits.
  \item Coordination capacity ratio:
        $\kappa_{\mathrm{USD}} \approx 0$.
\end{itemize}
\end{example}

\begin{corollary}[Hierarchy of Coordination Capacity]
\label{cor:hierarchy}
Under the above estimates:
\[
\kappa_{\mathrm{BTC}}
\;\gg\;
\kappa_{\mathrm{Gold}}
\;\gg\;
\kappa_{\mathrm{USD}}.
\]
Quantitatively:
\begin{align*}
\kappa_{\mathrm{BTC}} / \kappa_{\mathrm{Gold}}
  &\approx 10^{20}, \\
\kappa_{\mathrm{Gold}} / \kappa_{\mathrm{USD}}
  &\approx 10^{18}\text{--}10^{19}.
\end{align*}
The ratio $\kappa_{\mathrm{BTC}} / \kappa_{\mathrm{Gold}} \approx 10^{20}$
suggests Bitcoin represents a \emph{qualitative phase transition} in
coordination capacity, not merely an incremental improvement over
prior monetary technologies.
\end{corollary}


\subsubsection{Relation to Prior Definitions}

The coordination capacity ratio $\kappa$ complements the earlier
definitions in this section:

\begin{itemize}
  \item \textbf{Irreversibility density} $\rho_{\irr}$ (Definition 4.X)
        measures joules per unit transferred---a flow quantity.
  \item \textbf{Security exponent} $\beta$ (Definition 4.X) measures
        the exponential decay rate of reorganization probability.
  \item \textbf{Coordination capacity ratio} $\kappa$ measures the
        ratio of cumulative thermodynamic depth to monetary uncertainty---a
        stock quantity capturing the system's attractiveness as a
        coordination substrate.
\end{itemize}

These quantities are related: high $\rho_{\irr}$ contributes to high
$\mathcal{S}$ (cumulative irreversibility), while protocol ossification
contributes to low $\mathcal{H}$ (monetary predictability). Together,
they determine $\kappa$.

\begin{remark}[Why $\kappa$ Matters]
While $C_{\mathrm{coord}}$, $\rho_{\irr}$, and $\beta$ characterize
the \emph{operational} properties of a protocol (capacity, density,
security), $\kappa$ characterizes its \emph{attractiveness} as a
coordination substrate. The Capital Flow Conjecture (Conjecture
\ref{conj:capital-flow}) posits that $\kappa$ determines long-run
capital allocation across competing monetary systems.
\end{remark}